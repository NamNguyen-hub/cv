\documentclass[a4paper,11pt]{article}
%\usepackage[utf8]{inputenc}
%\usepackage{amsmath}
%\usepackage{amssymb}
%\usepackage{graphicx}
%\usepackage[top = 0.75in, bottom = 0.75in, left = 0.5in, right = 0.5in]{geometry}
%\usepackage{titlesec}
%\usepackage{enumerate}
%\usepackage{natbib}
%\usepackage{booktabs}
%\usepackage[pdftex, colorlinks = true,
 %           linkcolor = blue,
  %          urlcolor  = blue,
   %         citecolor = blue,
    %        anchorcolor = blue]{hyperref}
%\usepackage{caption}
%\usepackage{atbegshi}
%\usepackage{lscape}
%\usepackage{caption}
%\usepackage[multiple]{footmisc}
%\usepackage[title]{appendix}
%\usepackage{multicol}


\usepackage[utf8]{inputenc}
\usepackage{amsmath}
\usepackage{amssymb}
\usepackage{graphicx}
\usepackage[margin=1in]{geometry}
\usepackage{titlesec}
\usepackage{enumerate}
\usepackage{natbib}
\usepackage{booktabs}
\usepackage[pdftex, colorlinks = true,
            linkcolor = blue,
            urlcolor  = blue,
            citecolor = blue,
            anchorcolor = blue]{hyperref}
\usepackage{caption}
\usepackage{atbegshi}
\usepackage{lscape}
\usepackage{caption}
\usepackage[multiple]{footmisc}
\usepackage[title]{appendix}
\usepackage{multicol}

%%adding header to page with type of document and name
\usepackage{fancyhdr}
\pagestyle{fancy}
\fancyhf{}
\lhead{Eduard Storm}
\rhead{Research Statement}
\cfoot{\thepage}

\renewcommand{\headrulewidth}{0.4pt}
\renewcommand{\footrulewidth}{0.0pt}

\linespread{1.28}
%%change back to 1.30 sometime after editing
      
\setlength{\parindent}{1.5em}
\setlength{\parskip}{0.5em}

\oddsidemargin=-15pt % leftmargin is 1 inch
\textwidth=6.5in   % textwidth of 6.5in leaves 1 inch for right margin

\begin{document}
\thispagestyle{plain}
%\pagenumbering{gobble}

\begin{center}
 {\Large \textbf{Research Statement}} \\
 Eduard Storm \\
 \href{mailto:estorm@carleton.edu}{estorm@carleton.edu} $|$ \href{https://eduardstorm.com/}{eduardstorm.com}
\end{center}

\noindent

My primary research interests include Applied Microeconomics, especially Labor Economics and Economics of Inequality. In my dissertation ``Tasks, Skills, and Wages in Labor Markets" I utilize Survey data on Tasks performed at work to enhance our understanding of the concept of `Skill' and how it translates into Wage differences. In contrast to conventional methods, relying on formal qualifications such as education or experience, emphasis on workplace heterogeneity offers novel insight into skill differences and varying levels of productivity. In sum, I argue this framework is superior to traditional methods in explaining wage differences between and especially within groups of workers


In completed research I heavily utilize German Employment Surveys providing self-reported information on worker's job-related activities. In one working paper I compare this worker-level information with an online database, which is based on (commonly used) occupation-level data and assessment by labor market experts about occupation-specific requirements. Notably, I highlight the conceptual benefits of worker-level data in a simple model in which workers can earn efficiency gains by specializing in core tasks within their own occupation and validate this approach with various empirical tests. This insight is applied to the migration context in a different paper in which I decompose wage differences along the entire wage distribution. In this study I emphasize variation in interactive tasks as a key driver of the rising wage gap in Germany as native workers utilize their comparative advantage in interactive activities. My paper is the first to demonstrate that this specialization pattern extends occupational borders and thus has important implications for sources of imperfect substitutability of different types of labor in the production function and, from a policy perspective, the integration of immigrant workers. In 2020, this project was awarded the \textit{Richard Perlman Prize for Outstanding Paper in Labor Economics}, underscoring its substantial contribution to the literature.


Current work in progress explores the impact of technological change on wage inequality by combining task information from the Employment Surveys with  administrative data from the \textit{Institut f{\"u}r Arbeitsmarkt- und Berufsforschung}. Concentrating on the occupational dimension, this research examines varying degrees of task specialization in modern labor markets and how they affect the wage structure. In a related study I estimate the returns to the number of tasks workers perform at work. In a nutshell, my current projects add a twist to Adam Smith's classic insight on Gains from Specialization by using novel data to test if they truly apply to all workers. 



The first chapter of my dissertation, titled ``On the Measurement of Tasks: Does Expert-Data Get it Right?'', uses novel task data from Germany assembled by the German Institute for Vocational Education (BIBB), the Institute of Employment Research (IAB) and the Federal Institute of Occupational Safety and Health (BAuA). Commonly used data such as the Occuptional Information Network (O*Net) database in the US are based on occupational analysts and therefore provide external assessment on the job-specific task requirements at the occupation-level. In contrast, the BIBB/IAB \& BIBB/BAuA Employment Surveys I am using are unique in the sense that they provide self-reported information on job-related activities by individuals, thus task requirements at the worker level. Comparing this task data with a free online portal for occupations provided by the German Federal Employment Agency (FDZ) called ``Berufenet Expert Database'', similar in spirit to the O*Net database, my findings document substantial heterogeneity in task assignments at the individual level. Notably, I demonstrate that expert-data identifies main tasks by occupation only half the time at most. These results therefore indicate incomplete information on the part of external assessments about the task content at work and argue for utilizing worker-level information in labor market studies, whenever feasible. 


My job market paper ``The Native-Migrant Wage Gap Revisited: Evidence from Individual Task Data'' applies the detailed information on individual task assignments to explore the wage gap between native and foreign workers. 
In this study, I decompose wage differences along the wage distribution, adopting a statistical tool called `Recentered Influence Function' (RIF). This way I am able to estimate unique wage responses resulting from a change in job activities by nativity and at different points of the distribution. According to this distributional analysis, variation in interactive tasks has been a key contributor to the rising native-foreign wage gap, suggesting that native and foreign workers perform distinct activities at work. Importantly, variation in task assignments is most pronounced among high-wage earners, explaining up to 25\% of wage differences, and can also be found among workers with similar formal qualifications. Previous research has documented how natives utilize their comparative advantage in interactive tasks by choosing occupations intensive in communication-heavy activities. However, my research is the first to demonstrate that this specialization pattern can likewise be found within occupations and as this trend has become more meaningful in recent years it reinforced already existing wage disparities. My research thus has important implications for the integration of immigrant workers and offers a novel source of imperfect substitutability between native and foreign workers, which is at the core of small migration-induced wage effects usually found in the literature. 

The third chapter of my dissertation is still under progress and is tentatively titled ``Wage Inequality: A Task-Based Approach''. This paper seeks to fill a gap in the literature by offering a refined understanding on how variation in task assignments affects wage inequality. To accomplish this goal, I am combining task information from the BIBB/IAB \& BIBB/BAuA Employment Surveys with high-quality German administrative data on daily wages, spanning over 40 years, for which I had to request a confidential data use agreement. The current findings of my dissertation indicate substantial heterogeneity in task assignments between and within occupations. Borrowing the methodology from my second chapter, I decompose wage differences along the entire wage distribution over time, focusing on contributions from between- and within-occupation differences in task assignments. The former is calculated via occupation-level task measures. The latter is approximated by occupation-specific variability measures such as the interquartile range or the Gini coefficient. Concentrating on the occupational dimension, this research thus examines the degree of specialization in modern labor markets and how it affects the wage structure. Due to the computation-intensive effort of this exercise, results are still pending. 

%, favored by skill-biased technological change. Along with changes in labor market institutions such as declining unionization, this phenomenon is considered one of the key reasons for rising levels of wage inequality.

My research interests transcend the scope of my dissertation.  In future work I aim to combine task with linked employer-employee data to conflate research on the task approach with wage inequality at the firm level. Recent studies show how skilled workers increasingly work together in high-wage paying firms. I am eager to embed task information in a labor market search model to explore the relationship between rising importance of tasks and the rising assortativeness between high-wage workers and high-wage firms documented recently. In a similar project, I revisit the century-old question of Gains from Specialization. Analyzing trends in the number of tasks at work, I study whether workers responded differently to technological change in regards to the number of job-specific activities. Tentative results indicate that over time college graduates have become increasingly specialized in a set of core tasks. Workers with a high school diploma, on the other hand, tend to perform more tasks nowadays. One possibility is that high school graduates aim to make themselves indispensable by becoming all-rounders, for instance in light of rising levels of offshoring and outsourcing. In this regard, my research is related to questions studied in International Economics. After my graduation, I aspire to explore these issues more thoroughly in a general equilibrium model in which workers choose an optimal number of tasks and firms operate in an environment subject to skill-biased technological change.

Outside of the field of Labor Economics I am moreover interested in Economics of Education. For instance, in a current project outside of my dissertation I am investigating how immigration affects learning outcomes of native youth using German Census data. Finally, over my graduate career I have familiarized myself with Machine Learning and Big Data techniques and part of my interest lies in how these methods can inform and augment empirical methodology in Economics. I anticipate that the projects outlined above, which place emphasis on both theory and empirics, will remain as my primary focus of research for the near future. 




\end{document}
