\documentclass[a4paper,12pt]{article}
\usepackage[utf8]{inputenc}
\usepackage{amsmath}
\usepackage{amssymb}
\usepackage{graphicx}
\usepackage[margin=1in]{geometry}
\usepackage{titlesec}
\usepackage{enumerate}
\usepackage{natbib}
\usepackage{booktabs}
\usepackage[pdftex, colorlinks = true,
            linkcolor = blue,
            urlcolor  = blue,
            citecolor = blue,
            anchorcolor = blue]{hyperref}
\usepackage{caption}
\usepackage{atbegshi}
\usepackage{lscape}
\usepackage{caption}
\usepackage[multiple]{footmisc}
\usepackage[title]{appendix}
\usepackage{multicol}

%%adding header to page with type of document and name
\usepackage{fancyhdr}
\pagestyle{fancy}
\fancyhf{}
\lhead{Eduard Storm}
\rhead{Teaching Philosophy}
\cfoot{\thepage}

\renewcommand{\headrulewidth}{0.4pt}
\renewcommand{\footrulewidth}{0pt}

\linespread{1.3}
      
\setlength{\parindent}{1.5em}
\setlength{\parskip}{0.5em}

\oddsidemargin=-15pt % leftmargin is 1 inch
\textwidth=6.5in   % textwidth of 6.5in leaves 1 inch for right margin

\begin{document}
\thispagestyle{plain}

\begin{center}
 {\Large \textbf{Teaching Philosophy}} \\
 Eduard Storm \\
 \href{mailto:estorm@uwm.edu}{estorm@uwm.edu} %$|$ \href{http://eduardstorm.academic.bio/}{eduardstorm.academic.bio}
\end{center}
The beauty of teaching Economics is that your audience consists of applied economists - they just don’t know it yet. By opening student’s eyes about the scope of their daily decisions, they reflect on the allocation of their own scarce resources – the ultimate premise of Economics!  A key objective of my teaching principles is to instruct students on critical thinking. To achieve this goal, I frequently incorporate economic history to draw lessons from the past, introduce various international examples to offer novel insight, and create an active learning environment where students from diverse backgrounds can learn effectively. 

Applying a global perspective has been favored by the multicultural platform provided by the University of Wisconsin-Milwaukee (UWM) with international students from nearly 100 countries. Supplemented by a comprehensive curriculum, this environment enabled me to craft my teaching skills towards a diverse audience and across various fields. In `Principles of Macroeconomics', for example, we have drawn lessons from the Great Depression to understand how the global economy emerged comparably unscathed off the Financial Crisis 07/08. To facilitate complex comparisons, I like to incorporate pop-cultural elements such as watching the famous Bank Run scene from the movie \textit{It’s A Wonderful Life} and contrast this clip with statements from present-day politicians emphasizing cohesiveness and reminding citizens of institutional features aimed at protecting their savings. This way, students get a sense about the danger of a self-fulfilling prophecy in times of crisis and how global cooperation can mitigate harmful effects. While historical lessons have shown to be fruitful, it is equally important to incorporate contemporaneous policies. For instance, when I taught an intermediate course on `International Economic Relations', debates regarding the refugee crisis were omnipresent, in part with dubious arguments. Hence, we analyzed the impact of immigration in class using common trade models and supported by empirical data. This way, I can ensure that my students are well-informed and do not fall for spurious arguments brought up in public media discussions at times. 

To maximize learning outcomes, it is important to create a welcoming atmosphere where students are integrated in class discussions. To accomplish this goal, I employ a multi-layered strategy. First, start using everyday examples everyone can relate to. Second, after building up fundamentals, progress towards more complex, policy-oriented analysis. Third, ask open questions on real-life policies to stimulate critical thinking and inform students on current events. For example, I regularly pass out a sheet of paper in my introductory macroeconomic class, asking for student’s information on estimated consumption levels for different income levels. Using this input, I estimate a class-specific consumption function. The first exposure to a consumption function thus creates a personal connection. As students have been involved in the process, they are more eager to learn how this kind of information can be utilized by policymakers. Equipped with the necessary theoretical background, I can then go on and provide stylized facts on the US economy, asking students to discuss how they would finance a proposed infrastructure bill, for instance. The key insight of this type of exercise is that decision-making requires sacrifices. Critically reflecting on these trade-offs is the main skill I want students to take away from my classes. 


Discussing topics from multiple perspectives is essential given the diverse classrooms I have encountered at UWM. Teaching at a large public university involves dealing with a student body with various socio-economic characteristics, notably varying degrees of prior knowledge and almost 40\% of Undergraduates being First-Generation students. A matter close to my heart has been teaching `Economics of Personal Finance', both as a one-credit course and as part of a bridge program for incoming freshmen since Summer 2017.  The course is focused on practical topics such as budgeting and designed to assist minority students who are applying for a scholarship. As a First-Generation student myself, I appreciate feedback on how “informative and fun” students consider this class and its lifelong lessons. 

By repeatedly putting myself in someone else’s shoes, teaching has taught me to articulate ideas with clarity and creativity. This progress not only made me a better communicator, but also more empathetic as I learned about struggle from students from troubled regions. These conversations made me realize the classroom can be a getaway for all of us which is why I incorporate humorous elements such as poking fun at my own daily struggles. Big picture, teaching has made me a better person. It gave me a sense of responsibility I humbly accept, which is why I consider the William L. Holahan Prize for Outstanding Teaching in Economics in 2019 one of the greatest accomplishments of my young career. This accolade is based on student evaluations in a discipline I take great pride in. Moreover, I am appreciative of various students reaching out to me early on in my career and asking for recommendation letters, helping them to get into great programs such as UW Madison. By teaching the everyday value of Economics I like to believe that I do not only help them to become better students, but ultimately to become better citizens. 



\end{document}
