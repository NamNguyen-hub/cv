\documentclass[a4paper,12pt]{article}
\usepackage[utf8]{inputenc}
\usepackage{amsmath}
\usepackage{amssymb}
\usepackage{graphicx}
\usepackage[margin=1in]{geometry}
\usepackage{titlesec}
\usepackage{enumerate}
\usepackage{natbib}
\usepackage{booktabs}
\usepackage[pdftex, colorlinks = true,
            linkcolor = blue,
            urlcolor  = blue,
            citecolor = blue,
            anchorcolor = blue]{hyperref}
\usepackage{caption}
\usepackage{atbegshi}
\usepackage{lscape}
\usepackage{caption}
\usepackage[multiple]{footmisc}
\usepackage[title]{appendix}
\usepackage{multicol}

%%adding header to page with type of document and name
\usepackage{fancyhdr}
\pagestyle{fancy}
\fancyhf{}
\lhead{Eduard Storm}
\rhead{Teaching Philosophy}
\cfoot{\thepage}

\renewcommand{\headrulewidth}{0.4pt}
\renewcommand{\footrulewidth}{0pt}

%\linespread{1.3}
\usepackage{setspace}
%\doublespacing
\onehalfspacing
      
\setlength{\parindent}{1.5em}
\setlength{\parskip}{0.5em}

\oddsidemargin=-15pt % leftmargin is 1 inch
\textwidth=6.5in   % textwidth of 6.5in leaves 1 inch for right margin

\begin{document}
\thispagestyle{plain}

\begin{center}
 {\Large \textbf{Teaching Statement}} \\
 Eduard Storm \\
 \href{mailto:estorm@carleton.edu}{estorm@carleton.edu} $|$ \href{https://eduardstorm.com/}{eduardstorm.com}
\end{center}
The beauty of teaching Economics is that your audience consists of applied economists - they just don’t know it yet. By opening student’s eyes about the scope of their daily decisions, they reflect on the allocation of their own scarce resources – the ultimate premise of Economics!  A key objective of my teaching principles is to instruct students on critical thinking. To achieve this goal, I frequently incorporate economic history to draw lessons from the past, introduce various international examples to offer novel insight, and create an active learning environment where students from diverse backgrounds can learn effectively. 

Applying a global perspective has been favored by the multicultural platform provided by my Alma mater, the University of Wisconsin-Milwaukee (UWM), with international students from nearly 100 countries. Supplemented by a comprehensive curriculum, this environment enabled me to craft my teaching skills towards a diverse audience and across various fields. In `International Economic Relations', for example, we have drawn lesson from the Past to understand current global trade patterns. The invention of the automobile near Stuttgart still shapes Germany's industry and is closely associated with the label `Made in Germany'. And who knows if people would have even heard about `Silicon Valley' had it not been for a few exceptional entrepreneurs and proximity to Stanford University. Using various examples like these makes students aware about the value of History and how it shapes modern economies. 

While historical lessons have shown to be fruitful, it is likewise important to incorporate recent events. To challenge the common perception of Economics as a dry subject, it is incumbent to discuss contemporaneous topics so students see the diverse applications offered within the field. For instance, in my `Principles of Macroeconomics' class at Carleton College I frequently incorporate the impact of COVID-19 using basic economic models. Employing a simple framework such as the AD/AS model, students can actively elaborate on effects of Fiscal Policy in form of the \textit{CARES} Act or Monetary Policy in form of cuts in the Federal Funds Rate on economic output. Using news from the same day or week not only helps disprove the reputation of an out of touch  subject, it also creates stimulating discussions about economic projections, thus allowing students to use the fundamental insight they gained in prior classes to reflect about questions for which there are no definite answers yet. 

To maximize learning outcomes, it is important to create a welcoming atmosphere where students are integrated in class discussions, but subject to a clear and coherent structure. To accomplish this goal, I employ a multi-layered strategy. First, start using everyday examples everyone can relate to. Second, after building up fundamentals, progress towards more complex, policy-oriented analysis. Third, ask open questions on real-life policies to stimulate critical thinking and inform students on current events. For example, I regularly pass out a sheet of paper in my introductory macroeconomic class, asking for student’s information on estimated consumption levels for different income levels. Using this input, I estimate a class-specific consumption function. The first exposure to a consumption function thus creates a personal connection. As students have been involved in the process, they are more eager to learn how this kind of information can be utilized by policymakers. Equipped with the necessary theoretical background, I can then go on and provide stylized facts on the US or Global economy, asking students to discuss how they would finance a proposed stimulus package, for instance. The key insight of this type of exercise is that decision-making requires sacrifices. Critically reflecting on these trade-offs is the main skill I want students to take away from my classes. 


Discussing topics from multiple perspectives can be helpful to higlight said trade-offs and account for the diverse classrooms I have encountered over my career. For these reasons, I aim to take time throughout the semester to divide students into groups with the goal of facilitating vibrant discussions. After finishing the segment on GDP, for example,  students prepare short presentations in which they present alternative measures of economic well-being such as the Green GDP or HDI and follow these up with open Q\&A sessions. Not only do these conference-like exercises allow students to expand their horizon, they also gain insight into novel topics potentially of interest for young scholars like themselves. Importantly, open conversations like these allow a student body with various socio-economic characteristics to engage in a diverse set of intellectual curiosities. In this regard, one of my favorite subjects to teach has been `Economics of Personal Finance'. This course focuses on practical topics such as budgeting and is designed to assist First-Gens and minority students who are applying for a scholarship. As a First-Generation student myself, I appreciate feedback on how “informative and fun” students consider this class and its lifelong lessons. 

By repeatedly putting myself in someone else’s shoes, teaching has taught me to articulate ideas with clarity and creativity. This progress not only made me a better communicator, but also more empathetic as I learned about struggle from students from troubled regions. These conversations made me realize the classroom can be a getaway for all of us - now more so than ever before - which is why I incorporate humorous elements such as poking fun at my own daily struggles. Big picture, teaching has made me a better person. It gave me a sense of responsibility I humbly accept, which is why I consider the William L. Holahan Prize for Outstanding Teaching in Economics in 2019 one of the greatest accomplishments of my young career. This accolade is based on student evaluations in a discipline I take great pride in. Moreover, I am appreciative of various students reaching out to me early on in my career and asking for recommendation letters, helping them to get into great programs such as UW Madison. By teaching the everyday value of Economics I like to believe that I do not only help them to become better students, but ultimately to become better citizens. 



\end{document}
